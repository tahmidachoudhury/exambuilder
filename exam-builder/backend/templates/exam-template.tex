\documentclass[a4paper,12pt]{article}
\usepackage{amsmath, amssymb} % Math symbols 
\usepackage{graphicx} % For including images
\usepackage{enumitem} % For better list formatting
\usepackage{geometry} % Adjusts margins
\usepackage{times} % Use Times New Roman font
\usepackage{tcolorbox} % For adding border boxes 
\geometry{margin=0.4in} % 0.4-inch margin for a clean layout 

\begin{document} 
\title{\textbf{Mathematics Exam Paper}} % Bold title 
\date{} % No date 
\maketitle % Add a border around the entire page 

\begin{tcolorbox}
[width=\textwidth, height=\textheight, colframe=black, colback=white, arc=5mm, boxrule=0.2mm]

%question 1

\textbf{1} \hspace{1em} 
\[f(x) = (x + 3)(x + 2)(x - 1)\] 

\begin{enumerate}[label=(\alph*)]
\item Sketch the curve \( y = f(x) \), showing the points of intersection with the coordinate axes. \hfill \textbf{(3)}
\item Showing the coordinates of the points of intersection with the coordinate axes, sketch on separate diagrams the curves:
    \begin{enumerate}[label=(\roman*)]
    \item \( y = f(x - 3) \) \hfill \textbf{(2)}
    \item \( y = f(-x) \) \hfill \textbf{(2)}
    \end{enumerate}
\end{enumerate}

\hfill \textbf{(Total for question 39 is 7 marks)}

\hrule height 0.4pt
\vspace{1em}

%question 2

\textbf{2} \hspace{1em}

\textbf{Showing the coordinates of the points of intersection with the coordinate axes, sketch on separate diagrams the curves:}
\begin{enumerate}[label=(\alph*)]
\vfill
\item \( y = f(x) + 2 \) \hfill \textbf{(2)}
\item \( y = -f(x) \) \hfill \textbf{(2)}
\vfill
\end{enumerate}

\hfill \textbf{(Total for question 2 is 6 marks)}

\hrule height 0.4pt
\vspace{1em}

%question 3

\textbf{1} \hspace{1em} % Left-aligned question number
\textbf{$f(x) = (x + 3)(x + 2)(x - 1)$} 

\begin{enumerate}[label=(\alph*)] 
\item Sketch the curve \( y = f(x) \), showing the points of intersection with the coordinate axes. \hfill \textbf{(3)} 
\vfill % Fills available vertical space (adjust dynamically) 

\item Showing the coordinates of the points of intersection with the coordinate axes, sketch on separate diagrams the curves:
    \begin{enumerate}[label=(\roman*)]
    \item \( y = f(x - 3) \) \hfill \textbf{(2)} 
    \vfill 
    \item \( y = f(-x) \) \hfill \textbf{(2)} 
    \vfill
    \end{enumerate}
\end{enumerate}

\end{tcolorbox} % Close the bordered box 
\end{document}